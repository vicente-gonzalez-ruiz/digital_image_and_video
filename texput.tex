% Emacs, this is -*-latex-*-

\title{Digital Images and Videos}

\maketitle

\tableofcontents

\section{Images}
An image can be described as a
\href{https://en.wikipedia.org/wiki/Matrix_(mathematics)}{matrix} of
\href{https://en.wikipedia.org/wiki/Pixel}{pixels} (PIcture X-ray
ELementS), arranged in a regular 2D grid. Such sequences are
usully captured by a
\href{https://en.wikipedia.org/wiki/Charge-coupled_device}{CCD (Charge
  Coupled Device)}. In maths, an image can be denoted by
\begin{equation}
  {\mathbf I} = \{\{{\mathbf I}_{y,x}\}\in\mathbb{P}^{k}, (y,x)\in\mathbb{N}^2\},
  \label{eq:image}
\end{equation}
where $(y,x)$ is a spatial coordinate\footnote{Notice that we follow
the ``first row, then column'' notation to express the
coordinates. This is also used in NumPy~\cite{harris2020array}, for
example.}, and $\mathbb{P}^{k}$ represents the (pixels) \emph{values
domain}, with dimension (number of pixel components)
$k$~\cite{burger2016digital}.

\section{Videos}
In a similar way, a sequence of frames can be denoted by
\begin{equation}
  {\mathbf V} = \{{\mathbf V}_{t,y,x}\in\mathbb{P}^{k}, (t,y,x)\in\mathbb{N}^3\} = \{{\mathbf I}_t\},
  \label{eq:video}
\end{equation}
where $t$ represents time, and ${\mathbf I}$ is an image such
as the described in Eq.~\eqref{eq:image}.

Depending on $k$, we can have grayscale pixels ($k=1$), color pixels
($k=3$), and multispectral pixels ($k>3$).\footnote{When $k>3$ we also
use the term ``(super/hyper)-spectral'' pixels/frames/sequences, depending
on the value of $k$.} Finally, if we take into consideration the
number of bits/component that usually is $8$ (although this value can be larger
specially for grayscale frames), the pixel-depth will be $8$
bits/pixel for grayscale sequences and $3\times 8=24$ bits/pixel for
color ones.

\section{Resources}

\renewcommand{\addcontentsline}[3]{}% Remove functionality of \addcontentsline
\bibliography{python,image-processing,image-compression,text-compression}
